\documentclass[a4paper]{article}
 \usepackage[margin=1in]{geometry} 
\usepackage{amsmath,amsthm,amssymb,amsfonts}
\usepackage{graphicx} 

\newtheorem{theorem}{Theorem}
\newtheorem{definition}{Definition}

\newcommand{\negl}{\text{negl}}
\newcommand{\Enc}{\text{Enc}}
\newcommand{\Dec}{\text{Dec}}
\newcommand{\Gen}{\text{Gen}}
\newcommand{\Mac}{\text{Mac}}
\newcommand{\Vrfy}{\text{Vrfy}}
\newcommand{\Func}{\text{Func}}
\newcommand{\Perm}{\text{Perm}}
\newcommand{\Getvrfy}{\text{Get-Vrfy}}
\newcommand{\hash}{\text{Hash}}
\newcommand{\hashsec}{\text{Hash-sec}}
\newcommand{\hashprei}{\text{Hash-prei}}
\newcommand{\hashcoll}{\text{Hash-coll}}
\newcommand{\vrfyfile}{\text{Vrfy-file}}
\newcommand{\myn}{l_{in}(n)}
\newcommand{\C}{\mathcal{C}}
\newcommand{\K}{\mathcal{K}}
\newcommand{\F}{\mathcal{F}}
\newcommand{\h}{\mathcal{H}}
\newcommand{\V}{\mathcal{V}}
\newcommand{\A}{\mathcal{A}}
\newcommand{\OO}{\mathcal{O}}

\newenvironment{problem}[2][Problem]{\begin{trivlist}
\item[\hskip \labelsep {\bfseries #1}\hskip \labelsep {\bfseries #2.}]}{\end{trivlist}}
%If you want to title your bold things something different just make another thing exactly like this but replace "problem" with the name of the thing you want, like theorem or lemma or whatever
 
\begin{document}
 
%\renewcommand{\qedsymbol}{\filledbox}
%Good resources for looking up how to do stuff:
%Binary operators: http://www.access2science.com/latex/Binary.html
%General help: http://en.wikibooks.org/wiki/LaTeX/Mathematics
%Or just google stuff
 
\title{Chapter 06}
\author{Mingjia Huo}
\date{}
\maketitle

\begin{problem}{6.1}
\begin{enumerate}
    \item First 10 bits are $1,1,1,1,1,1,0,1,0,1$
    \item Compute the First 69 bits:\par
    1,1,1,1,1,1,0,1,0,1,0,1,1,0,0,\par
    1,1,0,1,1,1,0,1,1,0,1,0,0,1,0,\par
    0,1,1,1,0,0,0,1,0,1,1,1,1,0,0,\par
    1,0,1,0,0,0,1,1,0,0,0,0,1,0,0,\par
    0,0,0,1,1,1,1,1,1\par
    After 63 bits, the block becomes the same with initial block $(1,1,1,1,1,1)$. Thus it's maximal length.\par
\end{enumerate}
\end{problem}
\begin{problem}{6.6(b)}
Assume the key of the first and third round is $k_1$, and the key of second round is $k_2$.\par
Giving a single input/output pair $(x,y)$, fixing $k_1$:
\begin{itemize}
    \item Compute $x_1=x\oplus k_1$, this is the output of the first key-mixing step.
    \item Since we know the details of S-box substitution and mixing permutation, we can compute the output of the first round, denoted as $x_1'$.
    \item Using $k_1$, we can also compute $y\oplus k_1$ to be the value before the third key-mixing, denoted as $x_2'$.
    \item Given $x_2'$, compute the inverse of S-box substitution and mixing permutation in round two, denoted as $x_2$. 
    \item Thus $k_1=x_1'\oplus x_2$.
\end{itemize}
Thus, for each choice of $k_1$, there is only one possible $k_2$. So the attack can use time no more than $2^{64}$, much less than $2^{128}$.
\end{problem}
\begin{problem}{6.13}
\textbf{(a).}\par
Giving input/output pairs $(x,y),k_1\in\{0,1\}^n$, we have $y=F_{k_1}(F_{k_2}^{-1}(F_{k_1}(x)))$.\par
First, if $k_1$ is the valid key, we can compute $F_{k_1}^{-1}(y)=y_1$ in constant time. Then compute $x_1=F_{k_1}(x)$. Thus, with $y_1=F_{k_2}^{-1}(x_1)$, we can get all $k_2$ in constant time. Since there are totally three pairs, we can use the other two pairs of input/output values to verify whether $(k_1,k_2)$ is valid.\par
Given $x$, the output $y$ has $2^n$ choices, thus the error rate is aproximately $2^{-2n}$.\par
There are $2^n$ choices of $k_1$, thus the probability that we find the right keys is $(1-2^{-2n})^{2^{n}}\approx e^{2^{-n}}$, which is negligble to $n$.\par
Thus with high probability, we can recover the entire key.\par\vspace{3ex}

\textbf{(b).}\par
Denote that $K_m$ is the set of $k$ such that $F_{k}^{-1}(0^n)=m$. All $K_m$ forms the set $\K$.
In preprocessing, We can construct a table of $\{0,1\}^n\rightarrow \K$ by simply enumerate all $k$, which takes $2^n$ time.\par
$\forall k\in\{0,1\}^n$, compute $m=F_{k}^{-1}(0^n)$, then add $k$ to $K_m$, which saves in the table described above.\par
After preprocessing, given $m_2$, we can look up the table, find the line of $m_2$, and get the all keys which satisfies the condition in constant time.
\par\vspace{3ex}
\textbf{(c).}\par
\begin{enumerate}
    \item Compute $F_{k_1}^{-1}(0^n)$ and denote as $x$.
    \item Choose $x$ as input, get access to the encryption oracle, then we get $y$.
    \item Compute $F_{k_1}^{-1}(y)$ and denote as $y'$.
    \item Thus, $y'=F_{k_2}^{-1}(0^n)$. Use the method in problem (b) to get all $k_2$ in constant time.
\end{enumerate}
The procedure above takes constant time.
\par\vspace{3ex}

\textbf{(d).}\par
\begin{enumerate}
    \item Preprocessing as the method in problem (b). (roughly $2^n$ time.)
    \item Fixing $k_1$, run as the process in problem (c). 
    (When $k_1$ is fixed, the time of (c) is constant, and only need a single chosen inputs. Thus, this step needs roughly $2^n$ time and $2^n$ chosen inputs. )
    \item When we get a valid $(k_1,k_2)$, use another two input/output pairs to verify it. (The probability that a key pair is valid is roughly $2^{-n}$. Thus this step takes constant time and inputs.)
\end{enumerate}
To sum up, this attack needs roughly $2^n$ time and $2^n$ chosen inputs.
\par\vspace{3ex}
\end{problem}

\begin{problem}{6.19}
\textbf{(a).}\par
In ideal-cipher model, $F$ is a permutation.
Attack:\begin{enumerate}
    \item Randomly choose $k_1,k_2$.
    \item Ask the oracle $F_{k_1}^{-1}(0^n),F_{k_2}^{-1}(0^n)$, and get $x_1,x_2$.
    \item Then $(k_1,x_1)=(k_2,x_2)$ is a collision. That is $H(k_1,x_1)=F_{k_1}(x_1)=0^n=F_{k_2}(x_2)=H(k_2,x_2)$.
\end{enumerate}
\par\vspace{3ex}

\textbf{(b).}\par
\begin{proof}
Assume there are totally $q(n)$ queries asked by the adversary $\A$, and the length of hash value is $n$.\par
If $(k,x)$ are asked, then a hash value $h=F(k,x)\oplus x\oplus k$ can be computed. And if $F^{-1}(k,y)$ is asked, then the answer $x$ is returned, and $\A$ can compute $x\oplus y\oplus k$ and get the hash value $h$. \par 
Denote the hash values involved in the $q(n)$ queries as $h_1,h_2,\cdots,h_{q(n)}$, while the key/input/output are denoted as $(k_i,x_i,y_i),1\le i\le q(n)$.\par
A collision is there is $1\le j<i\le q(n)$, such that $h_i=h_j$. 
\begin{itemize}
    \item Fix $i>j$, consider the probability that $h_i=h_j$. 
    \item Since $j$ is asked earlier, we first get $h_j=F(k_j,x_j)\oplus x_j\oplus k_j$.
    \item In $i$th query, there are two cases when a collision happens:
    \begin{itemize}
        \item $F(k_i,x_i)$ is asked: Since $F$ is ideal-cipher, $F(k_i,x_i)$ can uniformly set to any $y\in\{0,1\}^n$, expect for values answered by $F(k_i,\cdot)$. Thus $F(k_i,x_i)$ equals to $h_j\oplus k_i\oplus x_i$ holds with probability no more than $1/(2^n-i_1)$. 
        \item $F(k_i,y_i)^{-1}$ is asked: Similarly, $x_i$ equals to $h_j\oplus k_i\oplus y_i$ with probability no more than $1/(2^n-i_1)$.  
    \end{itemize}
    Thus a collision $h_i=h_j$ happens with probability $<1/(2^n-(i-1))$. Since $i\le q(n)$, the probability is less than $\frac{1}{2^{n-1}}$.
\end{itemize}\par
Taking a union bound of all pairs of $(i,j),1\le j<i\le q(n)$, the collision rate $<\frac{q(n)^2}{2^n}$, which is negligible.
\end{proof}
\par\vspace{3ex}

\textbf{(c).}\par
Attact:\begin{enumerate}
    \item Randomly choose $k_1,k_2$.
    \item Ask the oracle $F_{k_1}^{-1}(k_1),F_{k_2}^{-1}(k_2)$, and get $x_1,x_2$.
    \item Then $(k_1,x_1)=(k_2,x_2)$ is a collision. That is $H(k_1,x_1)=F_{k_1}(x_1)\oplus k_1=k_1\oplus k_1=0^n=H(k_2,x_2)$.
\end{enumerate}
\end{problem}

\begin{problem}{6.21}
Assume $l=|x|\ge2|k|$.\par
A brute search takes roughly $2^{l/2}$ time when there is $\frac12$ probability to find a collision. \par
We can simply enumerate $k$, until we find the key that it's easy for it to find inputs $x$ for which $F_k(x)=x$. If $F_k(x_1)=x_1,F_k(x_2)=x_2$, then $h(k,x_1)=F_k(x_1)\oplus x_1=F_k(x_2)\oplus x_2=h(k,x_2)$, a collision. And it takes roughly $2^{|k|-1}$ time when there is $\frac12$ probability to find a collision. \par
Thus it's better than brute force approach.

\end{problem}
\end{document}

