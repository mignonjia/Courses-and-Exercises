\documentclass[a4paper]{article}
 \usepackage[margin=1in]{geometry} 
\usepackage{amsmath,amsthm,amssymb,amsfonts}
\usepackage{graphicx} 
\usepackage{algorithm,algorithmicx}
\usepackage{hyperref}
\usepackage{algpseudocode}
\usepackage{epsfig}

\newtheorem{theorem}{Theorem}
\newtheorem{definition}{Definition}

\newcommand{\poly}{\text{poly}}
\newcommand{\negl}{\text{negl}}
\newcommand{\Enc}{\text{Enc}}
\newcommand{\GenRSA}{\text{GenRSA}}
\newcommand{\Dec}{\text{Dec}}
\newcommand{\Gen}{\text{Gen}}
\newcommand{\Mac}{\text{Mac}}
\newcommand{\Vrfy}{\text{Vrfy}}
\newcommand{\eav}{\text{eav}}
\newcommand{\tsb}{\text{2sb}}
\newcommand{\ssb}{\text{sb}}
\newcommand{\isb}{\text{isb}}

\newcommand{\lsb}{\text{lsb}}
\newcommand{\half}{\text{half}}
\newcommand{\KE}{\text{KE}}
\newcommand{\Query}{\text{Query}}
\newcommand{\hash}{\text{Hash}}
\newcommand{\trans}{\text{trans}}
\newcommand{\Decaps}{\text{Decaps}}
\newcommand{\Encaps}{\text{Encaps}}
\newcommand{\Invert}{\text{Invert}}
\newcommand{\multi}{\text{multi}}
\newcommand{\KEM}{\text{KEM}}
\newcommand{\Pubk}{\text{Pubk}}
\newcommand{\C}{\mathcal{C}}
\newcommand{\U}{\mathcal{U}}
\newcommand{\Y}{\mathcal{Y}}
\newcommand{\K}{\mathcal{K}}
\newcommand{\F}{\mathcal{F}}
\newcommand{\h}{\mathcal{H}}
\newcommand{\V}{\mathcal{V}}
\newcommand{\Z}{\mathbb{Z}}
\newcommand{\G}{\mathbb{G}}
\newcommand{\A}{\mathcal{A}}
\newcommand{\OO}{\mathcal{O}}

\newenvironment{problem}[2][Problem]{\begin{trivlist}
\item[\hskip \labelsep {\bfseries #1}\hskip \labelsep {\bfseries #2.}]}{\end{trivlist}}
%If you want to title your bold things something different just make another thing exactly like this but replace "problem" with the name of the thing you want, like theorem or lemma or whatever
 
\begin{document}
 
%\renewcommand{\qedsymbol}{\filledbox}
%Good resources for looking up how to do stuff:
%Binary operators: http://www.access2science.com/latex/Binary.html
%General help: http://en.wikibooks.org/wiki/LaTeX/Mathematics
%Or just google stuff
 
\title{Chap 11}
\author{Mingjia Huo}
\date{}
\maketitle

\begin{problem}{11.3}$ $\par
(a).\par
Given $one$-$way$ public key encryption scheme (Gen,Enc,Dec) and randm oracle $H$, here is the construction:
\begin{enumerate}
    \item Gen: Run $\Gen(1^n)$ to obtain keys $(pk,sk)$.
    \item Encaps: Uniformly choose $m\in\{0,1\}^n$, compute $c=\Enc_{pk}(m)$. And $k=H(m)$.
    \item Decaps: Given $c,pk$, compute $k=H(\Dec_{pk}(c))=\Decaps_{pk}(c)$. 
\end{enumerate}\par
Then prove it's CPA-secure.\par
Consider the experiment $\KEM_{\A,\Pi}^{cpa}(n)$. For an adversary $\A$, let $\Query$ be the event that $\A$ queries $m$ to $H$. Then,
\begin{align*}
    \Pr[\KEM_{\A,\Pi}^{cpa}(n)=1]&=\Pr[\KEM_{\A,\Pi}^{cpa}(n)=1\land\Query]+\Pr[\KEM_{\A,\Pi}^{cpa}(n)=1\land\overline{\Query}]\\
    &\le \Pr[\Query]+\Pr[\KEM_{\A,\Pi}^{cpa}(n)=1\mid\overline{\Query}]\times\Pr[\overline{\Query}]\\
    &\le \Pr[\Query]+\Pr[\KEM_{\A,\Pi}^{cpa}(n)=1\mid\overline{\Query}]
\end{align*}
Since $H$ is a random oracle, $H(m)$ is uniformly distributed from the perspective of the adversary, which is the same with a uniform string. So
$$\Pr[\KEM_{\A,\Pi}^{cpa}(n)=1\mid\overline{\Query}]=\frac12.$$\par
As for $\Pr[\Query]$, we can prove it is negligible to $n$:\par
Use $\A$ to construct $\A'$ in the experiment of $one$-$way$ public key encryption scheme:
\begin{itemize}
    \item When $\A'$ is given $(pk,c)$, choose a uniform $k\in\{0,1\}^n$.
    \item Give $(pk,c,k)$ to $\A$ and then run $\A$. Record all $\A$'s queries to $H$, denote as set $Q$.
    \item Assume $|Q|=t$, uniform choose an item $m'\in Q$, and output $m'$.
\end{itemize}
If $\A$ queries on $m$, then $\A'$ has probability $\frac1t$ to output $m$. 
So $$\Pr[m'=m]\ge \frac1t\Pr[\Query].$$
Since $\Pr[m'=m]$ is negligible to $n$, $\Pr[\Query]$ is also negligible to $n$.\par
Thus $\Pr[\KEM_{\A,\Pi}^{cpa}(n)=1]\le\frac12+\negl(n)$. And this scheme is CPA-secure.\par\vspace{2ex}
(b).\par
Yes. Construct a scheme $\Pi=(\Gen,\Enc,\Dec)$ based on RSA. 
\begin{itemize}
    \item $\Gen(1^n)$: Run $\GenRSA(1^n)$ to obtain $N,e,d$. Here $N>2^n$.
    \item Enc: $c=m^e,m\in\{0,1\}^n$.
    \item Dec: $m=c^d$.
\end{itemize}\par
Assume RSA problem is hard relative to $\GenRSA$. Then we prove the scheme is one-way: Given $\A'$ for $\Pi$, construct $\A$ for $\text{RSA-inv}_{\A,\GenRSA}(n)$.\par
\begin{itemize}
    \item $\A'$ is given $(N,e,y)$.
    \item Run $\A(N,e,y)$, and get $m$.
    \item Output $m$.
\end{itemize}\par
Here,
\begin{align*}
    \negl(n)\ge&\Pr[\text{RSA-inv}_{\A,\GenRSA}(n)=1]\\
    =&\Pr[\text{One-Way}_{\A',\Pi}(n)=1\mid m\in\Z_N^*]\\
    =&(\Pr[\text{One-Way}_{\A',\Pi}(n)=1]-\Pr[\text{One-Way}_{\A',\Pi}(n)=1\land m\not\in\Z_N^*])\times(\Pr[m\in \Z_N^*])^{-1}\\
    \ge&(\Pr[\text{One-Way}_{\A',\Pi}(n)=1]-\Pr[m\not\in\Z_N^*])\times(\Pr[m\in \Z_N^*])^{-1}.
\end{align*}\par
Since $\phi(N)=(p-1)(q-1)>\frac12N>\frac12\cdot 2^n$, we have $(\Pr[m\in \Z_N^*])^{-1}\ge\frac12$. And we have $\Pr[m\not\in \Z_N^*]\le\negl(n)$.\par
Thus 
\begin{align*}
    \negl(n)\ge&(\Pr[\text{One-Way}_{\A',\Pi}(n)=1]-\Pr[m\not\in\Z_N^*])\times(\Pr[m\in \Z_N^*])^{-1}\\
    \ge& \frac12(\Pr[\text{One-Way}_{\A',\Pi}(n)=1]-\negl(n)).
\end{align*}\par
That is $$\Pr[\text{One-Way}_{\A',\Pi}(n)=1]\le3\negl(n).$$\par
So the experiment of one-way successes with probability negligible to $n$.
\end{problem}

\begin{problem}{11.7}
It's not CPA-secure. Give an attack of adversary $\A$:\par
\begin{itemize}
    \item When $\A$ is given $(\G,g,q,h)$, uniformly choose $r\in\Z_p$, output $(m_0,m_1)=(0,r)$.
    \item When given ciphertext $c$, if $c^q\equiv 1\mod p$, then output $0$; otherwise, output $1$.
\end{itemize}\par
Here $$\Pr[\Pubk_{\A,\Pi}^{cpa}(n)=1]=\frac12\Pr[\A(\G,g,q,h,c)=0\mid b=0]+\frac12\Pr[\A(\G,g,q,h,c)=1\mid b=1].$$\par
If $b=0$, then $c=h^r+m_0$, which is a quadratic residue. And $c^q\equiv1\mod p$, so the output of $\A$ is 0. Thus,
\[\Pr[\A(\G,g,q,h,c)=0\mid b=0]=1.\]\par
If $b=1$, then $c=h^r+m_1$, with probability $\frac {q+1}{p}$ to be a quadratic residue. So the probability that $c^q\not\equiv 1\mod p$ is $\frac{q}{p}>\frac14$.\par
Thus,
$$\Pr[\Pubk_{\A,\Pi}^{cpa}(n)=1]=\frac12\Pr[\A(\G,g,q,h,c)=0\mid b=0]+\frac12\Pr[\A(\G,g,q,h,c)=1\mid b=1]>\frac58.$$\par
So it's not CPA-secure.
\end{problem}

\begin{problem}{11.8}
(a).\par
Denote the bit of two parties as $a,b$, and assume $\Pr[a=0]=p$. Let the result be value $r\in\{0,1\}$.\par
Then $\Pr[r=0]=\Pr[a=0\land b=0]+\Pr[a=1\land b=1]=p\cdot\frac12+(1-p)\cdot\frac12=\frac12$. Also $\Pr[r=0]=\frac12$.\par
So the result is uniform.\par\vspace{2ex}
(b).\par
Assume $A$ outputs $(c_1,c_2)$.\par
If b wants the result to be 0, just output $(c_1g,c_2h)$. So the value it decrypts is the same as $b_A$.\par
If b wants the result to be 1, just output $(\frac{1}{c_1},\frac{g}{c_2})$. So the value it decrypts is $\frac{g}{g^{b_A}}=1\oplus b_A$.\par
\vspace{2ex}
(c).\par 
We should use a CCA-secure scheme.\par
\begin{definition}[secure coin flip protocol]
A secure coin flip protocol with public key encryption scheme $\Pi=(\Gen,\Enc,\Dec)$ should satisfy: $\forall$ adversary $\A$, 
$$\Pr[Pubk_{\A,\Pi}(n)=1]\le \frac12+negl(n).$$
\end{definition}\par\vspace{1ex}
Here we define the experiment $\Pubk_{\A,\Pi}(n)$ with the decryption oracle:
\begin{enumerate}
    \item Run $\Gen(1^n)$ to obtain $(pk,sk)$. Randomly choose $b\in \{0,1\}$ and compute $\Enc_{pk}(b)=c$. 
    Then uniformly choose the expected result bit $r\in \{0,1\}$.
    \item Give $(pk,c,r)$ and the oracle $\OO(\cdot)=\Dec_{sk}(\cdot)$ to $\A$. But $\A$ can't ask the oracle to decrypt $c$ directly. 
    \item $\A$ outputs $c'$. 
    \item If $b\oplus\Dec_{sk}(c')=r$, the experiment outputs $1$; otherwise, outputs $0$.
\end{enumerate}\par
We prove if $\Pi$ is a CCA-secure scheme, then the coin flip protocol is secure.\par
\begin{proof}
Construct an adversary $\A'$ for CCA-secure scheme based on $\A$ for coin flip protocol.\par
\begin{itemize}
    \item $\A'$ is given $pk$ and decryption oracle $\OO(\cdot)$.
    \item Output $(m_0,m_1)=(0,1)$ and get $c=\Enc_{pk}(m_b)$.
    \item Run $\A(pk,c,0)$, and get $c'$ from $\A$.
    \item Output the bit $b'=\OO(c')$.
\end{itemize}\par
In this construction, $\Pr[b=b']=\Pr[\Dec_{sk}(c)=\Dec_{sk}(c')]=\Pr[\A(pk,c,0)=1]$. Since $\Pi$ is CCA-secure, so $\Pr[b=b']\le\frac12+\negl(n)$. So $$\Pr[Pubk_{\A,\Pi}(n)=1]=\Pr[\A(pk,c,0)=1]\le\frac12+\negl(n).$$
\end{proof}
\end{problem}

\begin{problem}{11.13}
\begin{definition}[secure under $t$-multiple receivers]
A public key encryption scheme $\Pi=(\Gen,\Enc,\Dec)$ is secure under $t$-multiple receivers if $\forall \A$, $$\Pr[\Pubk_{\A,\Pi}^{t-\multi}(n)=1]\le\frac12+\negl(n).$$
Here, $t$ should be $\poly(n)$.
\end{definition}\par\vspace{2ex}
Here we define the experiment $\Pubk_{\A,\Pi}^{t-\multi}(n)$:\par
\begin{itemize}
    \item Run $\Gen(1^n)$ for $t$ times, and get $(pk_i,sk_i),1\le i\le t$. Choose uniform $b\in\{0,1\}$.
    \item Give $pk_i,i\in[t]$ to $\A$. Then $\A$ outputs $(m_0,m_1)$.
    \item Give $c=(\Enc_{pk_1}(m_b),\cdots,\Enc_{pk_t}(m_b))$ to $\A$.
    \item $\A$ outputs $b'$. If $b=b'$, output $1$; otherwise output $0$. 
\end{itemize}\par
We prove if $\Pi$ is a CPA-secure scheme, then it's secure under t-multiple receivers.\par
\begin{proof}
Construct an adversary $\A'$ for CPA-secure scheme experiment based on $\A$ for security under $t$-multiple receivers experiment.\par
\begin{itemize}
    \item $\A'$ is given $pk$. Uniform choose $r\in[t]$, and denote $pk$ as $pk_r$. 
    \item Run $\Gen(1^n)$ for $t-1$ times and get $(pk_i,sk_i)$, with $i\in[t],i\ne u$.
    \item Run $\A(pk_1, \cdots, pk_t)$ and then $\A$ outputs $(m_0,m_1)$.
    \item $\A'$ outputs $(m_0,m_1)$ and get $c$.
    \item Then $\A'$ computes $c_i=\Enc_{pk_i}(m_0), 1\le i<r$ and $c_i=\Enc_{pk_i}(m_1), r<i\le t$. Then give $c_i,i\in[t]$ to $\A$.
    \item Output the same bit as what $\A$ outputs, denoted as $b'$.
\end{itemize}\par
In this construction, $\Pr[Pubk_{\A',\Pi}^{cpa}(n)=1]=\Pr[b=b']=\Pr[\Dec_{sk}(c_r)=m_b]=\Pr[\A(c_1,\cdots,c_t)=b]$. \par
Denote the choice of $r$ as event $R$, we have
\begin{align*}
    &\Pr[\A(c_1,\cdots,c_t)=b]\\
    =&\sum_{r=1}^{t}\Pr[R=r]\times\Pr[\A(\Enc_{pk_1}(m_0),\cdots,\Enc_{pk_{r-1}}(m_0),\Enc_{pk_r}(m_b),\Enc_{pk_{r+1}}(m_1),\cdots,\Enc_{pk_t}(m_1))=b] \\
    =&\frac1t\sum_{r=1}^{t}\frac12\Pr[\A(\Enc_{pk_1}(m_0),\cdots,\Enc_{pk_{r-1}}(m_0),\Enc_{pk_r}(m_0),\Enc_{pk_{r+1}}(m_1),\cdots,\Enc_{pk_t}(m_1))=0]\\
    &+\frac1t\sum_{r=1}^{t}\frac12\Pr[\A(\Enc_{pk_1}(m_0),\cdots,\Enc_{pk_{r-1}}(m_0),\Enc_{pk_r}(m_1),\Enc_{pk_{r+1}}(m_1),\cdots,\Enc_{pk_t}(m_1))=1]\\
    =&\frac{1}{2t}(t-1+\Pr[\A(\Enc_{pk_i}(m_1))=1]+\Pr[\A(\Enc_{pk_i}(m_0))=0])\\
    =&\frac{1}{2t}(t-1+2\Pr[\Pubk_{\A,\Pi}^{t-\multi}(n)=1]).
\end{align*}\par
Since $\Pr[\A(c_1,\cdots,c_t)=b]=\Pr[Pubk_{\A',\Pi}^{cpa}(n)=1]\le\frac12+\negl(n)$, we have $$\frac{1}{2t}(t-1+2\Pr[\Pubk_{\A,\Pi}^{t-\multi}(n)=1]\le\frac12+\negl(n).$$ So $$\Pr[\Pubk_{\A,\Pi}^{t-\multi}(n)=1]\le\frac12+\negl(n).$$
\end{proof}
\end{problem}

\begin{problem}{11.15}
The algorithm is not CCA-secure, since it's deterministic.\par
The adversary $\A$ just gets the encryption of $(m_0,m_1)$, denoted as $(c_0,c_1)$. Then $\A$ outputs $(m_0,m_1)$. Compare the answer to $(c_0,c_1)$, he can know the value of $b$.
\end{problem}

\begin{problem}{11.20}$ $\par
\begin{algorithm}[h] 
\caption{Get\_value$(N,e,y)$: Return $x$ such that $x^e\equiv y\mod N$} 
\begin{algorithmic}[1] 
\State $LB\leftarrow 0$;
\State $UB\leftarrow N$;\ \   $//$ Lower bound and upper bound of x; 
\For{each $i\in [1,\log_2 N]$} 
\If {$\A(y)==0$}
\State $UB=(UB+LB)/2$; 
\Else
\State $LB=(UB+LB)/2$;
\EndIf
\State $y=(2^e)y$;
\If {There is only one number $x$ in $[LB,UB)$.}
\State Break;
\EndIf 
\EndFor 
\State \Return $x$; 
\label{code:recentEnd} 
\end{algorithmic} 
\end{algorithm}
The first $i$ iterations decide a value $t$ such that $\frac{t}{2^i}<x<\frac{t+1}{2^i},0\le t\le2^i-1$.\par
For iteration $i+1$:\par
\begin{itemize}
\item $0<2^ix\equiv x_i<\frac N2\mod{N}$: We have $2^{i+1}x\equiv 2x_i\mod N$. So $\lsb(2^{i+1}x\mod N)=\lsb(2x_i)=0$.
\item $\frac N2<2^ix\equiv x_i<N \mod N$: We have $2^{i+1}x\equiv 2x_i-N\mod N$. So $\lsb(2^{i+1}x\mod N)=\lsb(2x_i-N)=1$.
\end{itemize}
So $$\frac{2t}{2^{i+1}}<x<\frac{2t+1}{2^{i+1}}\Leftrightarrow \lsb(2^{i+1}x\mod N)=0,$$
$$\frac{2t+1}{2^{i+1}}<x<\frac{2t+2}{2^{i+1}}\Leftrightarrow \lsb(2^{i+1}x\mod N)=1.$$
Thus the algorithm can return $x$, such that $x^e\equiv y\mod N$.
\end{problem}

\begin{problem}{11.21}
Given $\A$ for experiment RSA-$\half_{\A,\GenRSA}(1^n)$, construct $\A'$ for RSA-$\lsb_{\A',\GenRSA}(1^n)$:
\begin{itemize}
    \item $\A'$ computes $t=2^e\mod N$.
    \item When $\A'$ is given $(N,e,y)$, compute $y'=yt\mod N$. So $x'=2x\mod N$. Give $(N,e,y')$ to $\A$.
    \item Output the same as what $\A$ outputs.
\end{itemize}\par
If $\half(x)=0$, then $y'=yt\equiv x^e\cdot 2^e\equiv(2x)^e\mod N$. Since $2x<N$, we have $\lsb(x')=0$.\par
If $\half(x)=1$, then $y'=yt\equiv x^e\cdot 2^e\equiv(x\cdot 2-N)^e\mod N$, where $0<2x-N<N$. Then $\lsb(x')=\lsb(2x-N)=1$.\par
So $$\half(x)=0\Leftrightarrow \lsb(x')=0,\ \ \ \half(x)=1\Leftrightarrow \lsb(x')=1.$$\par
Since $x$ is uniform in $\Z_N^*$, so $(2x\mod N)$ is also uniform in $\Z_N^*$.\par
Thus we have $$\Pr[\text{RSA-}\half_{\A,\GenRSA}(1^n)]=\Pr[\text{RSA-}\lsb_{\A,\GenRSA}(1^n)]\le \frac12+\negl(n).$$\par
So $\half(x)$ is also a hard-core prediction for the RSA problem.
\end{problem}

\end{document}

