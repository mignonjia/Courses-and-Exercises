\documentclass[12pt]{article}
 \usepackage[margin=1in]{geometry} 
\usepackage{amsmath,amsthm,amssymb,amsfonts}
\usepackage{graphicx} 

\newcommand{\privk}{\text{PrivK}}
\newcommand{\eav}{\text{eav}}
\newcommand{\out}{\text{out}}
\newcommand{\negl}{\text{negl}}
\newcommand{\Enc}{\text{Enc}}
\newcommand{\Dec}{\text{Dec}}
\newcommand{\Gen}{\text{Gen}}
\newcommand{\Mac}{\text{Mac}}
\newcommand{\Vrfy}{\text{Vrfy}}
\newcommand{\Func}{\text{Func}}
\newcommand{\Perm}{\text{Perm}}
\newcommand{\Bingo}{\text{Bingo}}
\newcommand{\Repeat}{\text{Repeat}}
\newcommand{\Newblock}{\text{NewBlock}}
\newcommand{\myn}{l_{in}(n)}

\newcommand{\C}{\mathcal{C}}
\newcommand{\A}{\mathcal{A}}
\newcommand{\OO}{\mathcal{O}}

\newenvironment{problem}[2][Problem]{\begin{trivlist}
\item[\hskip \labelsep {\bfseries #1}\hskip \labelsep {\bfseries #2.}]}{\end{trivlist}}
%If you want to title your bold things something different just make another thing exactly like this but replace "problem" with the name of the thing you want, like theorem or lemma or whatever
 
\begin{document}
 
%\renewcommand{\qedsymbol}{\filledbox}
%Good resources for looking up how to do stuff:
%Binary operators: http://www.access2science.com/latex/Binary.html
%General help: http://en.wikibooks.org/wiki/LaTeX/Mathematics
%Or just google stuff
 
\title{Chapter 04}
\author{Mingjia Huo}
\maketitle

\begin{problem}{4.1}
\begin{proof}
Consider an adversary $\A$ for a message authentication code $\Pi=(\Gen,\Mac,\Vrfy)$, such that the $\text{Mac-forge}_{\A,\Pi}(n)$ works in the following procedure:
\begin{itemize}
    \item A key $k$ is generated by running $\Gen(1^n)$.
    \item The adversary $\A$ is given input $1^n$ then randomly select a tag $t$ with uniform distribution. (For the length of tags is $t(n)$, each tag will be selected with probability $2^{-t(n)}$.) Then $\A$ uniformly draws $m$ from the message space and outputs $(m,t)$.
    \item Because $\A$ doesn't make any query to the oracle $\Mac_k(\cdot)$, $\A$ succeeds if and only if $\Vrfy_k
(m,t)=1$, which means $\text{Mac-forge}_{\A,\Pi}(n)=1$. 
\end{itemize}
Assume when we select $m$ with uniform distribution, the probability of tag $t$ is $\Pr[T=t]$. (Here $T$ is a random variable.) Then we compute:
\begin{align*}
    &\Pr[\text{Mac-forge}_{\A,\Pi}(n)=1]\\
    =&\sum_{t}\Pr[T=t]\times\Pr[t\text{ is selected by }\A]\\
    =&2^{-t(n)}\sum_t\Pr[T=t]\\
    =&2^{-t(n)}.
\end{align*}
If $t(n)=\mathcal{O}(\log n)$, we have $2^{-t(n)}=\mathcal{O}(\frac{1}{n^d})$, which is not $\negl(n)$. \par
So $t$ must be super-logarithmic.
\end{proof}
\end{problem}

\begin{problem}{4.6}
The algorithm is not secure.\par
\begin{proof}
Construct an adversary $\A$:
\begin{enumerate}
    \item $\A$ generates three different messages $m_1,m_2,m_3$ with length $n-1$.
    \item Then $\A$ gets access to oracle $\Mac_k(\cdot)$ and the oracle tells him $t_a=\Mac_k(m_0||m_1)$ and $t_b=\Mac_k(m_1||m_2)$. 
    \item $\A$ combines the first half of $\Mac_k(m_0||m_1)$ with the second half of $\Mac_k(m_1||m_2)$, then he get $t'$. 
    \item Output $(m_0||m_2, t')$.
\end{enumerate}
Assume the length of output value is $l_{out}(n)$ in the pseudorandom function $F$, and $F_k(0\|m_0)=t_0,F_k(0\|m_1)=t_1,F_k(1\|m_1)=t_2,F_k(1\|m_2)=t_3$. By the definition of $\Mac$, we have $$t_a=\Mac_k(m_0||m_1)=F_k(0\|m_0)\|F_k(1\|m_1)=t_0\|t_2,$$
$$t_b=\Mac_k(m_1||m_2)=F_k(0\|m_1)\|F_k(1\|m_2)=t_1\|t_3.$$
Then 
$$t'=\Mac_k(m_0||m_2)=F_k(0\|m_0)\|F_k(1\|m_2)=t_0\|t_3.$$
We construct $\Vrfy(m_0\|m_2,t')=1$ with $m=m_0\|m_2$ wasn't queried by $\A$ with probability 1, so the algorithm is not secure.
\end{proof}
\end{problem}

\begin{problem}{4.12}
Advantages: \par
The modification changes length $l$ to a single bit $0$ or $1$. The advantage is obvious: \begin{itemize}
    \item It shortens the length of message by $\mathcal{O}(\log n)$ bits and has better performance in practice.
    \item Without using the length, we can make encryption while reading the message.
\end{itemize}\par 
\vspace{2ex}
\begin{proof}
Assume $\A$ is a probabilistic polynomial-time adversary, $\Pi$ is the MAC for arbitrary-length messages, and $\Pi'$ is a MAC for fixed-length messages.\par
Just like the proof in textbook, we define:
\begin{itemize}
    \item Repeat: The same random identifier $r_i=r_j$ appear in two of
the tags returned by the MAC oracle.
    \item NewBlock:  At least one of the blocks $r\|j\|i\|m_i,j=0,1$ was never previously authenticated by the oracle.
\end{itemize}
So 
\begin{align*}
    &\Pr[\text{Mac-forge}_{\A,\Pi}(n)=1]\\
    =&\Pr[\text{Mac-forge}_{\A,\Pi}(n)=1\land\Repeat]\\
    +&\Pr[\text{Mac-forge}_{\A,\Pi}(n)=1\land\overline{\Repeat}\land\Newblock]\\
    +&\Pr[\text{Mac-forge}_{\A,\Pi}(n)=1\land\overline{\Repeat}\land\overline{\Newblock}].
\end{align*}
We show the three part is negligible respectively:
\begin{enumerate}
    \item $\Pr[\text{Mac-forge}_{\A,\Pi}(n)=1\land\Repeat]\le\Pr[\Repeat]$. And the probability of event Repeat is exactly the probability that $r_i=r_j$ for some $i\ne j$. Applying birthday bound in the textbook, we have $\Pr[\Repeat]\le\frac{q(n)^2}{2^{n/4}}$. Since $\A$ makes only polynomially many queries, this value is negligible.
    \item $\Pr[\text{Mac-forge}_{\A,\Pi}(n)=1\land\overline{\Repeat}\land\Newblock]\le \Pr[\text{Mac-forge}_{\A,\Pi}(n)=1\land\Newblock]$. It is also negligible by $Claim\ 10$ in the textbook.
    \item Finally, we prove that $$\Pr[\text{Mac-forge}_{\A,\Pi}(n)=1\land\overline{\Repeat}\land\overline{\Newblock}]=0.$$
    Let $q=q(n)$ denote the number of MAC oracle queries made by $\A$, and $r_i$ denote the denote the random identifier used to answer the $i$th oracle query. Let $(m,t)$ be the output of $\A$. \par
    Because there is no NewBlock, we have $r\in\{r_1,\cdots,r_q\}$. Assume $r=r_j$, and the $j$th query is about $m^{(j)}$, which is represented by $d'$ blocks. \par 
    Consider the total block number $d$ of $(m,t)$ output by $\A$:
    \begin{enumerate}
        \item $d=d'$: If Mac-forge$_{\A,\Pi}(n)=1$, then we must have $m\ne m^{(j)}$.  Since $m$ and $m^{(j)}$ have equal length, there must be at least one index $i$ for which $m_i\ne m^{(j)}_i$. \par
        Since the $i$th block is $r\|0\|i\|m_i$ with $i$ represents the position of block, it was then never authenticated in the $j$th Mac query. So this is a NewBlock, a contradiction.
        \item $d\ne d'$: Consider the last block $r\|1\|d\|m_d$. In the $j$th Mac query, there is only one block $r\|1\|d'\|m^{(j)}_{d'}$ which has ``1'' to represent it's the last block. But $d\ne d'$, so $r\|1\|d\|m_d$ is a NewBlock, a contradiction.\par
    \end{enumerate}\par
    Thus $$\Pr[\text{Mac-forge}_{\A,\Pi}(n)=1\land\overline{\Repeat}\land\overline{\Newblock}]=0.$$
\end{enumerate}
To sum up, 
\begin{align*}
    &\Pr[\text{Mac-forge}_{\A,\Pi}(n)=1]\\
    \le&\negl(n)+\negl(n)+0\\
    =&\negl(n).
\end{align*}
So it's secure.
\end{proof}
\par
\end{problem}

\begin{problem}{4.25}
Firstly, since $F$ is a permutation, so $\A$ just randomly chooses $c\in\{0,1\}^{n}$, there is always a $r$ and $m$, such that $\Enc(r\|m)=c$, and $\Dec(c)=m$. So
\[\Pr[\Enc\text{-Forge}_{\A,\Pi}(n)=1]=1, \]
which means this scheme $\Pi$ is not unforgeable and is not a authenticated encryption scheme.\par\vspace{2ex}
Next we prove it's CCA-secure.\par

\begin{proof}
Without loss of generality, assume the permutation is fixed length 
Given any adversary $\A$, we can construct a distinguisher $D$ which access an oracle $\OO$:$\{0,1\}^{n}\rightarrow\{0,1\}^{n}$ and $\OO^{-1}$. In detail:
\begin{enumerate}
    \item Run $\A(1^{2n})$. 
    \item When $\A$ queries its encryption oracle on a message $m\in \{0,1\}^{n}$, answer this query in the following way:
    \begin{enumerate}
        \item choose uniform $r\in\{0,1\}^n$.
        \item Query $\OO(r\|m)$ and obtain response $y$.
        \item Return the ciphertext $y$ to $\A$.
    \end{enumerate}
    \item When $\A$ queries its decryption oracle on a ciphertext $c\in \{0,1\}^{n}$, answer this query in the following way: compute $\OO^{-1}(c)$ and the second half is $m$. 
    \item When $\A$ outputs messages $m_0,m_1\in\{0,1\}^n$, choose a uniform bit $b\in\{0,1\}$ and then:
    \begin{enumerate}
        \item choose uniform $r_0\in\{0,1\}^n$.
        \item Query $\OO(r_0\|m_b)$ and obtain response $y$.
        \item Return the ciphertext $y$ to $\A$.
    \end{enumerate}
    \item Continue answering encryption and decryption oracle queries of $\A$ as before until $\A$ outputs a bit $b'$. Output $1$ if $b'=b$, and $0$ outherwise.
\end{enumerate}

Thus, $D$ outputs $1$ \textit{if and only if} $\A$ succeeds. Let $\Pi$ denotes our construction, and $\widetilde{\Pi}$ denotes the theme when we replace $F_k$ with a uniform permutation $f\in\Perm_n$. Then:
\[\Pr[\privk_{\A,\Pi}^{cca}(2n)=1]=\Pr_{k\leftarrow\{0,1\}^{2n}}[D^{F_k(\cdot),F_k^{-1}(\cdot)}(1^{2n})=1],\]
\[\Pr[\privk_{\A,\widetilde{\Pi}}^{cca}(2n)=1]=\Pr_{f\leftarrow\Perm_{2n}}[D^{f(\cdot),f^{-1}(\cdot)}(1^{2n})=1].\]
By the definition of pseudorandom permutation, we have 
\[\mid\Pr[D^{F_k(\cdot),F_k^{-1}(\cdot)}(1^{2n})=1]-\Pr[D^{f(\cdot),f^{-1}(\cdot)}(1^{2n})=1]\mid\le\negl(2n).\]
Thus, to prove $\Pr[\privk_{\A,\Pi}^{cca}(n)=1]\le\frac12+\negl(n),$ we only need to prove:\[\Pr[\privk_{\A,\widetilde{\Pi}}^{cca}(n)=1]\le\frac12+\negl(n)\]\par
\vspace{3ex}

Let $\C$ be the set of all ciphertext that has been asked by $\A$ or answer by the oracle. That is, $\Enc(m_{ask}) = c$ or $Dec(c)$ is asked. Let \textbf{Repeat} be the event that when $\A$ output $(m_0,m_1)$, the ciphertext $c^* = \Enc(m_b)$ is in $\C$. So,
\begin{align*}
    &\Pr[\privk_{\A,\widetilde{\Pi}}^{cca}(n)=1]\\
    =&\Pr[\privk_{\A,\widetilde{\Pi}}^{cca}(n)=1\land\Repeat]+\Pr[\privk_{\A,\widetilde{\Pi}}^{cca}(n)=1\land\overline{\Repeat}]
\end{align*}
Separately:
\begin{enumerate}
    \item $\Pr[\privk_{\A,\widetilde{\Pi}}^{cca}(n)=1\land\Repeat]$:\par
    There are two cases:
    \begin{enumerate}
        \item There is some $m$ asked by $\A$, such that $\Enc(m)$ outputs $c^*$. \par
        If $m\ne m_b$, the probability is $0$; and if $m=m_b$, the probability is equal to the probability that their $r$ are equal. There are at most $q(n)$ queries, so the probability is $\le\frac{q(n)}{2^{n/2}}=\negl(n)$.
        \item $\A$ asks $c$ before the experiment outputs $(m_0,m_1)$. Since there are $2^{-n/2}$ values that $\Enc(m_b)$ have, so the probability is also $\negl(n)$.
    \end{enumerate}
    To sum up, $\Pr[\privk_{\A,\widetilde{\Pi}}^{cca}(n)=1\land\Repeat]=\negl(n)$.
    \item $\Pr[\privk_{\A,\widetilde{\Pi}}^{cca}(n)=1\land\overline{\Repeat}]$:\par
    Let's first have a deep insight to the queries. \par We know that $f\in\Perm_n$. Then a query can delete some impossible functions and save the possible ones. When there is no \textbf{Repeat}, the encryption and decryption can be viewed as the same: to match a ciphertext $c\in\{0,1\}^{n}$ to a message $m\in\{0,1\}^{n/2}$. \par
    After $q(n)$ queries, assume there are $n_0$ ciphertexts which matches $m_0$ and $n_1$ ciphertexts which matches $m_1$. And the number of $r$ they used is also $n_0$ and $n_1$, denotes as sets $R_0,R_1$. \par
    So there are $2^{n/2}-n_0$ ciphertexts waiting to be matched to $m_0$, and $2^{n/2}-n_1$ ciphertexts waiting to be matched to $m_1$.\par
    
    Since $f$ is uniformly drawn from permutations, so given $r$ not involved in the queries of $m_0,m_1$, we have $$\Pr[f(r\|m_0)=c^*] = \Pr[f(r\|m_1)=c^*] = p.$$
    The probability taken over uniform chosen of functions $f$ which satisfy $\A$'s queries and the output of $c^*=\Enc(m_b)$.
    Furthermore, $$\sum_{r\not\in R_0}\Pr[f(r\|m_0)=c^*] = (2^{n/2}-n_0)\times p$$
    $$\sum_{r\not\in R_1}\Pr[f(r\|m_1)=c^*] = (2^{n/2}-n_1)\times p$$
    Thus, when $c=\Enc(m_b)$, 
    \begin{align*}
        &\Pr[\Dec(c)=m_0\land\overline{\Repeat}]\\
        =&\Pr[\Dec(c)=m_0\mid\overline{\Repeat}]\times\Pr[\overline{\Repeat}]]\\
        =&\frac{\sum_{r\not\in R_0}\Pr[f(r\|m_0)=c^*]}{\sum_{r\not\in R_0}\Pr[f(r\|m_0)=c^*]+\sum_{r\not\in R_1}\Pr[f(r\|m_1)=c^*]}\times(1-\negl(n))\\
        =&\frac{2^{n/2}-n_0}{2^{n/2+1}-n_0-n_1}\times(1-\negl(n))\\
        \le&\frac12 +\negl(n)
    \end{align*}
    The last inequality holds because $n_0,n_1$ are polynomial of $n$. \par
    Similarly, $\Pr[\Dec(c)=m_1\land\overline{\Repeat}]]\le\frac12 +\negl(n)$. So whatever $b'$ that $\A$ outputs, \begin{align*}
        &\Pr[\privk_{\A,\widetilde{\Pi}}^{cca}(n)=1\land\overline{\Repeat}]\\
        =&\Pr[\Dec(c)=m_0\land\overline{\Repeat}]\Pr[b=0]+\Pr[\Dec(c)=m_1\land\overline{\Repeat}]\Pr[b=1]\\
        \le&\frac12 +\negl(n)
    \end{align*}
\end{enumerate}
\par\vspace{2ex}
To sum up, \begin{align*}
    &\Pr[\privk_{\A,\widetilde{\Pi}}^{cca}(n)=1]\\
    =&\Pr[\privk_{\A,\widetilde{\Pi}}^{cca}(n)=1\land\Repeat]+\Pr[\privk_{\A,\widetilde{\Pi}}^{cca}(n)=1\land\overline{\Repeat}]\\
    \le&\negl(n)+\frac12 +\negl(n)\\
    =&\frac12 +\negl(n),
\end{align*}
which finishes the proof of CCA-secure.
\end{proof}
\end{problem}

\begin{problem}{10}
First, give $F_k'$ as:
		\begin{equation*}  
			F_k'(m)= \begin{cases}
					0^{|m|}, & m=k \\  
					F_k(k), & m=F_k^{-1}(0^{|m|}) \\
					F_k(m), & others
				\end{cases} 
		\end{equation*}
Since $F_k$ is a permutation, then we exchange two matches in $F_k$ to get $F_k'$, so $F_k'$ \textbf{is also a permutation.}\par\vspace{2ex}
Construct $D$ which can access oracle $\OO.\OO^{-1}$:
\begin{enumerate}
    \item Given $1^n$, then $\A$ simply asks $\OO^{-1}(0^n)$. 
    \item Assume the answer is $k$. Then uniformly select a message $m\in\{0,1\}^n$, and queries $\OO(m)=c$.
    \item If $c=F_k(m)$, output $1$; otherwise, output 0.
\end{enumerate}\par
If $\OO=f$, then $\Pr[{D}^{f(\cdot),f^{-1}(\cdot)}(1^n)=1]=\frac{1}{2^n}=\negl(n)$.\par
If $\OO=F_k'$, then $\Pr[{D}^{F_k'(\cdot),F_k'^{-1}(\cdot)}(1^n)=1]\le1-\frac{2}{2^n}=1-\negl(n).$\par
Thus,
\[\mid\Pr[{D}^{f(\cdot),f^{-1}(\cdot)}(1^n)=1]-\Pr[{D}^{F_k'(\cdot),F_k'^{-1}(\cdot)}(1^n)=1]\mid>\negl(n).\]
\textbf{So the theme is not a strong pseudorandom permutation.}\par\vspace{3ex}
Next, we prove that this theme is a \textbf{pseudorandom permutation.}\par
\begin{proof}
Prove by contradiction: Assume $D'$ can use oracle $\OO'$ to distinguish $F'$ from random function $f$, that is
\[\mid\Pr[{D'}^{F_k'(\cdot)}(1^n)=1]-\Pr[D'^{f(\cdot)}(1^n)=1]\mid>\negl(n).\]
Without loss of generation, let \[\Pr[{D'}^{F_k'(\cdot)}(1^n)=1]-\Pr[D'^{f(\cdot)}(1^n)=1]>\negl(n).\]
Let $Bingo$ be the event that \textbf{at least one of $k$ and $m_0=F_k^{-1}(0^{|m|})$ have been asked by $D'$}. Assume there are $q(n)$ queries.
Compute
\begin{align*}
    &\Pr[{D'}^{F_k(\cdot)}(1^n)=1]\\
    =&\Pr[{D'}^{F_k(\cdot)}(1^n)=1\land\Bingo]+\Pr[{D'}^{F_k(\cdot)}(1^n)=1\land\overline{\Bingo}]\\
    \le &\Pr[{D'}^{F_k(\cdot)}(1^n)=1\mid\Bingo]\times \Pr[\Bingo]+\Pr[{D'}^{F_k(\cdot)}(1^n)=1\land\overline{\Bingo}]\\
    \le&1\times \Pr[\Bingo]+\Pr[{D'}^{F_k(\cdot)}(1^n)=1\land\overline{\Bingo}]
\end{align*}
\textbf{First}, we prove $\Pr[\Bingo]=\negl(n)$.\par
If there is a $\mathcal{PPT}$ $A'$, such that $\Pr[\Bingo]>\negl(n)$, construct $D''$ with oracle $\OO$:
\begin{enumerate}
    \item Run $1^n$. Run $A'$.
    \item If $A'$ asks $m$, then $A$ compute $F_m(m)=c'$ and asks $\OO(m)=c$.
    \item If $c=0^n$ or $c=c'$, output $1$ and return. If not, give $c$ to $A'$.
    \item If $c=0^n$ or $c=c'$ don't happen in all the queries and $A'$ ends, uniformly output $b\in\{0,1\}$.
\end{enumerate}\par
If $\OO=f$, $\Pr[c=0^{n}\lor c=c']\le 2\times\frac{q(n)}{2^n}=\negl(n)$. So $\Pr[D''^{f(\cdot)}(1^n)=1]\le\frac12+\negl(n)$.\par
If $\OO=F_k$, then when $A'$ asks $k$, then $c=c'$ happens; and when $A'$ asks $m_0$, $c=0^{n}$ happens. (If $D''$ doesn't return, it means $\Bingo$ doesn't happen in $A'$.) So $$\Pr[c=0^{n}\lor c=c']\ge\Pr[\Bingo]>\negl(n),$$which means $\Pr[D''^{F_k(\cdot)}(1^n)=1]>\frac12+\negl(n)$.
Thus\[\Pr[D''^{F_k(\cdot)}(1^n)=1]-\Pr[D''^{f(\cdot)}(1^n)=1]>\negl(n),\]
which contradicts that $F_k$ is a pseudorandom permutation.\par
So $\Pr[\Bingo]=\negl(n)$.\par\vspace{4ex}

Use the conclusion above, if \[\Pr[{D'}^{F_k'(\cdot)}(1^n)=1]-\Pr[D'^{f(\cdot)}(1^n)=1]>\negl(n),\]
then \[\Pr[{D'}^{F_k(\cdot)}(1^n)=1\land\overline{\Bingo}]-\Pr[D'^{f(\cdot)}(1^n)=1]>\negl(n).\]\par
\vspace{2ex}
Construct $D$ based on $D'$ with oracle $\OO$. 
\begin{enumerate}
    \item Given $1^n$. Run $D$ the same as $D'$.
    \item When $D'$ asks to encrypt a message $m$, run $\OO(m)=c$ and give $c$ to $D'$.
    \item Output the same value with $D$.
\end{enumerate}
Analysis:\begin{enumerate}
    \item If $\OO=f$, $D$ and $D'$ behave the same. So
    \[ \Pr[D'^{f(\cdot)}(1^n)=1]=\Pr[D^{f(\cdot)}(1^n)=1] \]
    \item If $\OO=F_k$, 
    \begin{align*}
        &\Pr[{D}^{F_k(\cdot)}(1^n)=1]\\
        \ge &\Pr[{D}^{F_k(\cdot)}(1^n)=1\land\overline{\Bingo}]\\
        =&\Pr[{D'}^{F_k'(\cdot)}(1^n)=1\land\overline{\Bingo}],
    \end{align*}
\end{enumerate}
So 
\begin{align*}
    &\Pr[{D}^{F_k(\cdot)}(1^n)=1]-\Pr[D^{f(\cdot)}(1^n)=1]\\
    >&\Pr[{D}^{F_k(\cdot)}(1^n)=1\land\overline{\Bingo}]-\Pr[D'^{f(\cdot)}(1^n)=1]\\
    =&\Pr[{D'}^{F_k(\cdot)}(1^n)=1\land\overline{\Bingo}]-\Pr[D'^{f(\cdot)}(1^n)=1]\\
    >&\negl(n),
\end{align*}
a contradiction.\par
To sum up, $F_k'$ is a \textbf{pseudorandom permutation}.
\end{proof}
\end{problem}
\end{document}