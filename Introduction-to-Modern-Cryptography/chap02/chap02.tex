\documentclass[12pt]{article}
 \usepackage[margin=1in]{geometry} 
\usepackage{amsmath,amsthm,amssymb,amsfonts}
 
\newcommand{\N}{\mathbb{N}}
\newcommand{\Z}{\mathbb{Z}}
\newcommand{\K}{\mathcal{K}}
\newcommand{\M}{\mathcal{M}}
\newcommand{\C}{\mathcal{C}}
\newcommand{\Enc}{\text{Enc}}
\newcommand{\Dec}{\text{Dec}}
\newcommand{\Gen}{\text{Gen}}

\newenvironment{problem}[2][Problem]{\begin{trivlist}
\item[\hskip \labelsep {\bfseries #1}\hskip \labelsep {\bfseries #2.}]}{\end{trivlist}}
%If you want to title your bold things something different just make another thing exactly like this but replace "problem" with the name of the thing you want, like theorem or lemma or whatever
 
\begin{document}
%\renewcommand{\qedsymbol}{\filledbox}
%Good resources for looking up how to do stuff:
%Binary operators: http://www.access2science.com/latex/Binary.html
%General help: http://en.wikibooks.org/wiki/LaTeX/Mathematics
%Or just google stuff
 
\title{Chapter 02}
\author{Mingjia Huo}
\date{}
\maketitle

\begin{problem}{2.3} 
\textbf{Refute}: 
Assume $\mathcal{M}=\{0,1\}$ with uniform distribution, define (Gen, Enc, Dec):
\begin{itemize}
    \item $\Gen$: $\varnothing\rightarrow\K$, where $\mathcal{K}=\{k_1k_2\mid k_1\in\{0,1\}, k_2\in\{0,1,2\}\}, \Pr[K=k_1k_2]=\frac16$.
    \item $\Enc$: $\K\times\M\rightarrow\C$, where $\mathcal{C}=\{c_1c_2\mid c_1,c_2=0,1\}$. Here, $c_1=k_1\oplus m$, and
    \begin{equation*}
        c_2=\begin{cases}
        0, & k_2=0,1 \\
        1, & k_2=2
        \end{cases}
    \end{equation*}
    \item $\Dec$: $\K\times\C\rightarrow\M$, where $m=c_1\oplus k_1$.
\end{itemize}\par
In this construction, $\Pr[M=m\mid C=c]=\frac 12=\Pr[M=m]$. But $$\Pr[C=00]=\Pr[c_1=0]\times\Pr[c_2=0]=\frac13,$$ $$\Pr[C=01]=\Pr[c_1=0]\times\Pr[c_2=1]=\frac16,$$ a contradiction.
\end{problem}


\begin{problem}{2.11} 
\textbf{Part 1}: 
Assume $\mid\C\mid=\mid\M\mid=n$, $\mid\K\mid=l$, there exists a encryption scheme (Gen,Enc,Dec). Let $\M=\{m_1,\cdots,m_n\},\K=\{k_1,\cdots,k_l\},\C=\{c_1,\cdots,c_n\}$, and\begin{itemize}
    \item Gen: $\{0,1\}^*\rightarrow\K$
		\begin{equation*}  
			\Pr[\Gen=k_j]= \begin{cases}
					\frac 1n, &  j\in\{1,\cdots,l-1\} \\  
					\frac {n-l+1}{n}, & j=l  
				\end{cases} 
		\end{equation*}
    \item Enc: $\K\times\M\rightarrow\C$\par
    For $j\in\{1,\cdots,l-1\}$, $\Enc_{k_j}(m_i)=c_{(i+j\mod n)}$.\par
    For $j=l$, and $\forall t\in\{l,l+1,\cdots,n\}$, $\Pr[\Enc_{k_l}(m_i)=c_{(i+t \mod n)}]=\frac {1}{n-l+1}$.
    \item Dec: $\K\times\C\rightarrow\M$ \par
    For $j\in\{1,\cdots,l-1\}$, $\Dec_{k_j}(c_i)=c_{(i-j\mod n)}$.\par
    For $j=l$, and $\forall t\in\{l,l+1,\cdots,n\}$, $\Pr[\Dec_{k_l}(m_i)=c_{(i-t \mod n)}]=\frac {1}{n-l+1}$.
\end{itemize}\par

Then $\forall i,j\in[n],m_i\in\M$,
\begin{itemize}
    \item when $j\in\{(i+1)\mod n,\cdots, (i+l-1)\mod n\}:$ \par $$\Pr[\Enc_{K}(m_i)=c_j]=\Pr[\Enc_{k_{(j-i \mod n)}}(m_i)=c_j]=\frac1n;$$
    \item otherwise, $$\Pr[\Enc_{K}(m_i)=c_j]=\Pr[\Enc_{k_l}(m_i)=c_j]\times\Pr[K=k_l]=\frac{1}{n-l+1}\frac{n-l+1}{n}=\frac 1n,$$
\end{itemize}
which means $\forall m,m'\in\M, c\in\C,$
\[\Pr[\Enc_{K}(m)=c]=\Pr[\Enc_{K}(m')=c]=\frac1n.\]
So the construction ensures perfect secrecy.\par
And for message $m_i$, when
\begin{enumerate}
    \item $k\in\{k_1,\cdots,k_{l-1}\}$, then $\Dec_k(\Enc_k(m))=m$.
    \item $k=k_l$, assume $C_m=\{c\mid c\in\C, \Pr[\Enc_{k_l}(m)=c]\ne 0\}$, then
    \begin{align*}
        \Pr[\Dec_k(\Enc_k(m))=m]
        &=\sum_{c\in C_m}\Pr[\Enc_{k_l}(m)=c]\Pr[\Dec_{k_l}(c)=m]\\
        &=(n-l+1)\times\frac{1}{n-l+1}\times\frac{1}{n-l+1}\\
        &=\frac {1}{n-l+1}.
    \end{align*}
    \end{enumerate}
    To sum up, \begin{align*}
        &\Pr[\Dec_K(\Enc_K(m))=m]\\
        =&\sum_{k=k_1}^{k_{l-1}}\Pr[K=k]\times\Pr[\Dec_k(\Enc_k(m))=m]+\Pr[K=k_l]\times\Pr[\Dec_{k_l}(\Enc_{k_l}(m))=m] \\
        =&(l-1)\times\frac1n+\frac{n-l+1}{n}\times\frac{1}{n-l+1}\\
        =&\frac {l}{n}.
    \end{align*}
    Let $l=\lceil2^{-t}n\rceil$, so when $n$ is large enough, we have $l<n$, which means $\mid\K\mid<\mid\M\mid$. Then $$\Pr[\Dec_K(\Enc_K(m))=m]=\frac{\lceil2^{-t}n\rceil}{n}\ge2^{-t},$$ which satisfies the condition.
    \vspace{3ex}
    
    \textbf{Part 2:}\par
    From \textbf{Part 1} we have $\mid\K\mid=l=\lceil2^{-t}n\rceil$. Then we proof $l$ must $\ge \lceil2^{-t}n\rceil$.\par
    By condition, $\Pr[\Dec_K(\Enc_K(m))=m]\ge 2^{-t}$. So $$\sum_m\Pr[\Dec_K(\Enc_K(m))=m]\ge 2^{-t}n$$.
    Let $C_{k,m}=\{c\mid \Pr[\Enc_k(m)=c]\ne 0\}$. Then we have \begin{align*}
        &\sum_m\Pr[\Dec_K(\Enc_K(m))=m]\\
        =&\sum_m\sum_k\Pr[K=k]\Pr[\Dec_k(\Enc_k(m))=m]\\
        =&\sum_{m,k}\Pr[K=k]\sum_{c\in C_{k,m}}\Pr[\Enc_k(m)=c]\times\Pr[\Dec_k(c)=m\mid \Enc_k(m)=c]\\
        =&\sum_{m,k}\sum_{c\in C_{k,m}}\Pr[K=k]\Pr[\Enc_k(m)=c]\Pr[\Dec_k(c)=m\mid \Enc_k(m)=c]
    \end{align*}\par
    When $\Pr[\Enc_k(m)=c]\ne0$, $\Pr[\Dec_k(c)=m\mid \Enc_k(m)=c]=\Pr[\Dec_k(c)=m]$. And when $c\not\in C_{k,m}$, we have $\Pr[\Enc_k(m)=c]=0$. Thus,
    \begin{align*}
    &2^{-t}n\\
    \le&\sum_m\Pr[\Dec_K(\Enc_K(m))=m]\\
    =&\sum_{m,k}\sum_{c\in C_{k,m}}\Pr[K=k]\Pr[\Enc_k(m)=c]\Pr[\Dec_k(c)=m\mid \Enc_k(m)=c]\\
    =&\sum_{m,k}\sum_{c\in C_{k,m}}\Pr[K=k]\Pr[\Enc_k(m)=c]\Pr[\Dec_k(c)=m]\\
    =&\sum_{m,k,c}\Pr[K=k]\Pr[\Enc_k(m)=c]\Pr[\Dec_k(c)=m]
    \end{align*}\par
    On the other hand, by perfect secrecy, $$\forall m,m'\in\M,\Pr[\Enc_K(m)=c]=\Pr[\Enc_K(m')=c]=\Pr[C=c].$$\par
    Thus, $$\sum_{c,m}\Pr[C=c]\Pr[\Dec_k(c)=m]=\sum_{c}\Pr[C=c]\times1=1.$$ \par
    Obviously, $$\forall m,\Pr[C=c]=\sum_{k'\in\K}\Pr[K=k']\Pr[\Enc_{k'}(m)=c].$$\par Then,
    \begin{align*}
        l=&\sum_{k\in\K}1\\
        =&\sum_{k}\sum_{c,m}\Pr[C=c]\Pr[\Dec_k(c)=m]\\
        =&\sum_{k,c,m}\sum_{k'\in\K}\Pr[K=k']\Pr[\Enc_{k'}(m)=c]\Pr[\Dec_k(c)=m]\\
        =&\sum_{k=k',c,m}\Pr[K=k']\Pr[\Enc_{k'}(m)=c]\Pr[\Dec_k(c)=m]\\
        &+\sum_{k,c,m}\sum_{k'\ne k}\Pr[K=k']\Pr[\Enc_{k'}(m)=c]\Pr[\Dec_k(c)=m]\\
        \ge&\sum_{k=k',c,m}\Pr[K=k']\Pr[\Enc_{k'}(m)=c]\Pr[\Dec_k(c)=m]\\
        =&\sum_{k,c,m}\Pr[K=k]\Pr[\Enc_{k}(m)=c]\Pr[\Dec_k(c)=m]\\
        \ge&2^{-t}n.
    \end{align*}
    So the lower bound is $\lceil2^{-t}\mid\M\mid\rceil$.
\end{problem}
\end{document}